\documentclass[times, utf8, zavrsni]{fer}
\usepackage{booktabs}

\begin{document}

\thesisnumber{731}

\title{Proceduralno generiranje niza povezanih prostorija}

\author{Ana Marija Devčić}

\maketitle

% Ispis stranice s napomenom o umetanju izvornika rada. Uklonite naredbu \izvornik ako želite izbaciti tu stranicu.
\izvornik

% Dodavanje zahvale ili prazne stranice. Ako ne želite dodati zahvalu, naredbu ostavite radi prazne stranice.
\zahvala{}

\tableofcontents

\chapter{Uvod}
    Proceduralna generacija se često primjenjuje u razvoju video igara zbog svoje nepredvidivosti. Jedna od njenih primjena je u generiranju nivoa, to jest prostorija i terena u kojima se igra odvija. Proceduralno generirani nivoi su česti u \textit{roguelike} žanru igara, koji je karakterističan po tome da se može igrati beskonačno puno puta i igra je svaki put drugačija. Tradicionalno se u igrama nivoi stvaraju ručno, bili oni ručno nacrtani ili 3D modelirani. Ručno stvaranje nivoa ipak ima neke prednosti prema proceduralnoj generaciji te se stoga još uvijek koristi.

    Za levele je bitno da ne budu monotoni kako igraču ne bi bili dosadni. Pri ručnom izrađivanju nivoa dizajner zna kako ga učiniti zanimljivim. Moglo bi se reći da je \textit{level design} umjetnost sama po sebi. Proceduralna generacija, s druge strane, često dovodi do monotonije i toga da sve izgleda nasumično razbacano, zato što i jest. Potrebno je naći neki balans između efikasnosti i ponovne uporabljivosti proceduralne generacije te prirodnosti i umjetničke vrijednosti ručno dizajniranih nivoa. Kako bi se postiglo prirodno generiranje nivoa potrebno je proučiti postojeće tehnike proceduralne generacije te ih primjeniti na vlastito rješenje. Za testiranje će biti potrebno implementirati nekoliko parametara koji utječu na generaciju levela kako bi se mogli usporediti, te kako bi se moglo procjeniti što daje najbolje rezultate.



\chapter{Zaključak}
Zaključak.

\bibliography{literatura}
\bibliographystyle{fer}

\begin{sazetak}
Sažetak na hrvatskom jeziku.

\kljucnerijeci{Ključne riječi, odvojene zarezima.}
\end{sazetak}

% TODO: Navedite naslov na engleskom jeziku.
\engtitle{Title}
\begin{abstract}
Abstract.

\keywords{Keywords.}
\end{abstract}

\end{document}
